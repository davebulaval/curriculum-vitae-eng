\documentclass[10pt, oneside]{article}

% Aesthetics and colors
%-----------------------------------------------
\usepackage[hmargin = 1.25cm, vmargin = 1.5cm]{geometry} 
\usepackage[utf8]{inputenc}  % for the portuguese characters 
\usepackage[none]{hyphenat} % no hyphenated words
\usepackage[usenames,dvipsnames]{xcolor} 
\usepackage{microtype} % to optimize spacing
\usepackage{array}
\usepackage{lmodern}
\usepackage{tikz} % For photo placement
\usepackage{fancyhdr}
\pagestyle{fancy}
\fancyhf{}
\newcommand\tab[1][1cm]{\hspace*{#1}}
\usepackage{multirow}
\usepackage{longtable}

% Fonts and tweaks for XeLaTeX
%-----------------------------------------------
\usepackage{fontspec,xltxtra}
\usepackage{xltxtra}
\usepackage{xunicode}
\defaultfontfeatures{Mapping=tex-text}
%\setsansfont[Scale=MatchLowercase,Mapping=tex-text]{Gill Sans} % Font my name at the top


%%% Couleurs
\definecolor{baseline-gray}{HTML}{f0f2f2}
\definecolor{baseline-black}{HTML}{212222}
\definecolor{baseline-blue}{HTML}{2d3bff}
\definecolor{baseline-red}{HTML}{ff5043}

% url color and font
%-----------------------------------------------
\usepackage[colorlinks,urlcolor=baseline-black,bookmarks=false,hypertexnames=true]{hyperref}
\urlstyle{same} 

% Customized section headers
%-----------------------------------------------
\usepackage{titlesec} % Allows creating custom \section's
\titleformat{\section}{\bf\color{baseline-black}
	\scshape\Large\raggedright}{}{0em}{}[\color{black}\titlerule]
\titlespacing{\section}{0pt}{0pt}{5pt} % Spacing around titles
\renewcommand{\headrulewidth}{0pt} % Get rid of the default rule in the header

%contact info
\newcommand{\cvemail}{david.beauchemin.5@ulaval.ca}
\newcommand{\cvphonenumber}{(514) 250-3616}
\newcommand{\cvlinkedin}{https://www.linkedin.com/in/david-beauchemin/}
\newcommand{\github}{https://github.com/davebulaval}

\newcommand{\guillemet}[1]{\guillemotleft #1 \guillemotright}

% Customized subsection headers
%-----------------------------------------------
\titleformat{\subsection}{\bfseries\large\raggedright}{}{0em}{}[\color{black}]
\titlespacing{\subsection}{0pt}{2.5pt}{2.5pt}

% Settings for the vertical timeline
%-----------------------------------------------
\newcolumntype{L}{>{\raggedleft}p{0.14\textwidth}}
\newcolumntype{R}{p{0.8\textwidth}}
\newcommand\VRule{\color{baseline-gray}\vrule width 0.5pt}

\usepackage{fontawesome}

%---------------------------------------------------------
% ----- Begginning of the documment ---------------
%---------------------------------------------------------
\begin{document}
	
	% Name in the top center
	%-----------------------------------------------
	\par{\centering{\bf\sffamily\Huge David Beauchemin}\\\vspace{2pt}
		
		
		% Colored boxes with personal info
		%----------------------------------------------- 
		
		\par{\centering
			\href{mailto:\cvemail}{\faEnvelopeO} \,\, \href{mailto:\cvemail}{\cvemail}\\
			\faPhone \,\, \cvphonenumber \\
			\href{\github}{\faGithub}\,\, \href{\github}{Voir GitHub} \\
			\href{\cvlinkedin}{\faLinkedinSquare}\,\, \href{\cvlinkedin}{Voir profil}\\
		}
		
		\vspace{10pt}
		
		% Formal education
		%-----------------------------------------------
		\section*{Études}
		
		\begin{tabular}{L!{\VRule}R}
			2020 -- \tab[.7cm] & \textbf{Doctorat en informatique | Apprentissage automatique | Traitement automatique du langage naturel}\\
			\multirow{1}{*}{\includegraphics[scale=0.1]{images/UL_P.pdf} \tab[0.6cm]} &  Université Laval \\
			& \href{https://graal.ift.ulaval.ca/}{GRAAL - Groupe de recherche en apprentissage automatique de l'Université Laval}\\
			&\\[-6pt]
			
			2018 -- 2020           & \textbf{Maîtrise en informatique | Apprentissage automatique | Traitement automatique du langage naturel}\\
			\multirow{1}{*}{\includegraphics[scale=0.1]{images/UL_P.pdf} \tab[0.6cm]}  &  Université Laval \\
			& \href{https://graal.ift.ulaval.ca/}{GRAAL - Groupe de recherche en apprentissage automatique de l'Université Laval}\\
			& Sujet mémoire: \textit{Détection de doublons parmi des informations non structurées provenant de sources de données externes}\\
			&\\[-6pt]
			
			2015 -- 2018           & \textbf{Baccalauréat en actuariat} \\
			\multirow{1}{*}{\includegraphics[scale=0.1]{images/UL_P.pdf} \tab[0.6cm]}  &  Université Laval
		\end{tabular}
		
		\vspace{10pt}
		
		% Experience
		%-----------------------------------------------
		\section*{Expériences}
		
		\subsection{Professionnelles}
		\begin{tabular}{L!{\VRule}R}
			2019 -- \tab[.7cm] &{\bf {Expert en intelligence artificielle et vice-président des opérations}}\\
			\multirow{1}{*}{\includegraphics[scale=0.075]{images/Baseline.png} \tab[1.1cm]}					&Baseline\\
			& Recherche appliquée de solution en IA pour des clients \& formations sur mesure en IA\\
			&\\[-6pt]
			2019 -- \tab[.7cm] &{\bf {Fondateur, recherchiste, et intervieweur}}\\
			\multirow{1}{*}{\includegraphics[scale=0.025]{images/OpenLayer.png} \tab[0.8cm]}& \href{https://www.youtube.com/channel/UCB3tYpZ1ojiqAroyDN05Cyw}{OpenLayer podcas}t\\
			& Création de contenu en IA \\ 
			&\\[-6pt]
			2018 -- \tab[.7cm]   &{\bf {Organisation d'événements}}\\
			\multirow{1}{*}{\includegraphics[scale=0.025]{images/meetup.png} \tab[0.8cm]}& Meetup Machine Learning Québec\\
			& Co-organisation des différents événements à Québec \\ 
			&\\[-6pt]
			2019 -- 2020   &{\bf {Développeur}}\\
			\multirow{1}{*}{\includegraphics[scale=0.15]{images/poutyne.png} \tab[0.18cm]}& \href{https://poutyne.org/}{Poutyne}\\
			& Développement de fonctionnalité \\ 
			&\\[-6pt]
			2019 -- 2020  &{\bf {Scientifique IA}}\\
			\multirow{1}{*}{\includegraphics[scale=0.09]{images/Vooban.png} \tab[1.15cm]}& Vooban\\
			& Conseiller technique en intelligence artificielle
		\end{tabular}
		
		\subsection{Enseignements}
		\begin{tabular}{L!{\VRule}R}
			2016 -- \tab[.7cm] &{\bf {Auxiliaire de cours}}\\
			\multirow{1}{*}{\includegraphics[scale=0.1]{images/UL_P.pdf} \tab[0.6cm]}  &  Université Laval, École d'actuariat \\
			& Introduction à l'actuariat I, Gestion du risque financier I, Analyse et traitement collectif du risque, Informatique pour Laval University, Méthodes numériques en actuariat, Mathématiques actuarielles IARD II, Modèles linéaires en actuariat, Programmation avec R pour l'analyse de données, Correction des rapports de stage \\ 
			&\\[-6pt]
			2019 -- 2020&{\bf {Auxiliaire d'enseignement}}\\
			\multirow{1}{*}{\includegraphics[scale=0.1]{images/UL_P.pdf} \tab[0.6cm]}  &  Université Laval, CRDM - Centre de recherche en données massives \\
			& École d'hiver en apprentissage automatique
		\end{tabular}
		
		\subsection{Recherches et supervision de stagiaire}
		\begin{tabular}{L!{\VRule}R}
			2020 -- \tab[.7cm] &{\bf {Auxiliaire de recherche}}\\
			\multirow{1}{*}{\includegraphics[scale=0.1]{images/UL_P.pdf} \tab[0.6cm]}  &  Université Laval, IID - Institut intelligence et données\\
			& Supervision de deux stagiaires dans le projet \guillemet{femmes face aux défis de la transformation numérique : une étude de cas dans le secteur des assurances} \\ 
			& Christian Gagné, Centre des compétences futures\\ 
			&\\[-6pt]
			2020 &{\bf {Auxiliaire de recherche}}\\
			\multirow{1}{*}{\includegraphics[scale=0.1]{images/UL_P.pdf} \tab[0.6cm]}  &  Université Laval, École d'actuariat \\
			& Supervision d'un stagiaire et développement du plan d'infrastructure pour le projet de \guillemet{plateforme libre de validation et d'étalonnage de prévisions financières et actuarielles}\\ 
			& Vincent Goulet, Chaire d'actuariat de l'Université Laval\\ 
			&\\[-6pt]
			2019 &{\bf {Auxiliaire de recherche}}\\
			\multirow{1}{*}{\includegraphics[scale=0.1]{images/UL_P.pdf} \tab[0.6cm]}  &  Université Laval, GRAAL - Groupe de recherche en apprentissage automatique de l'Université Laval \\
			& Supervision d'un stagiaire  et développement d'une solution permettant la \guillemet{fusion d'enregistrements de risques commerciaux} \\ 
			& Luc Lamontagne, Chaire de recherche industrielle CRSNG - Intact Corporation financière sur l'apprentissage automatique en assurance\\ 
			&\\[-6pt]
			2019 &{\bf {Auxiliaire de recherche}}\\
			\multirow{1}{*}{\includegraphics[scale=0.1]{images/UL_P.pdf} \tab[0.6cm]}  &  Université Laval, GRAAL - Groupe de recherche en apprentissage automatique de l'Université Laval \\
			& Apprentissage de taxonomie de compétences professionnelles \\ 
			& Luc Lamontagne, Projet ENGAGE\\ 
			&\\[-6pt]
			2018 &{\bf {Auxiliaire de recherche}}\\
			\multirow{1}{*}{\includegraphics[scale=0.1]{images/UL_P.pdf} \tab[0.6cm]}  &  Université Laval, École d'actuariat \\
			& Mesures reliées à l'indexation conditionnelle dans un régime de retraite à prestations déterminées \\ 
			& Louis Adam, Chaire d'actuariat de l'Université Laval\\ 
			&\\[-6pt]
			2018 &{\bf {Auxiliaire de recherche}}\\
			\multirow{1}{*}{\includegraphics[scale=0.1]{images/UL_P.pdf} \tab[0.6cm]}  &  Université Laval, GRAAL - Groupe de recherche en apprentissage automatique de l'Université Laval \\
			& \textit{Weighted bootstrapping method for word relation extraction}  \\ 
			& Luc Lamontagne
		\end{tabular}
	
		\vspace{10pt}
		
		\section*{Publications}
		\subsection*{\hspace{.5cm} Article}
		\begin{tabular}{L!{\VRule}R}
	2022 & Vincent Primpied, David Beauchemin, and Richard Khoury. \href{https://caiac.pubpub.org/pub/iaeeogod}{\textit{Quantifying French Document Complexity}}. Proceedings of the Canadian Conference on Artificial Intelligence.\\
	&\\[-6pt] 
	2022 & David Beauchemin, Julien Laumonier, Yvan Le Ster, and Marouane Yassine. \href{https://caiac.pubpub.org/pub/72bhunl6}{\textit{``FIJO'': a French Insurance Soft Skill Detection Dataset}}. Proceedings of the Canadian Conference on Artificial Intelligence.\\
	&\\[-6pt] 
	2021 & Marouane Yassine, David Beauchemin, François Laviolette et Luc Lamontagne. \href{https://arxiv.org/abs/2112.04008}{\textit{Multinational Address Parsing: A Zero-Shot Evaluation}}. Accepted in the International Journal of Information Science and Technology (iJIST)\\
	&\\[-6pt] 	
	2021 & Marouane Yassine, David Beauchemin, François Laviolette et Luc Lamontagne. \href{https://arxiv.org/abs/2006.16152}{\textit{Leveraging Subword Embeddings for Multinational Address Parsing}}. IEEE CiSt, MNLP. \\
	&\\[-6pt] 		
	2020 & David Beauchemin, Nicolas Garneau, Eve Gaumond, Pierre-Luc Déziel, Richard Khoury et Luc Lamontagne. \href{https://arxiv.org/abs/2011.12183}{\textit{Generating Intelligible Plumitifs Summaries: Use Case Application with Ethical Considerations}}. The 13th International Conference on Natural Language Generation. \\
				&\\[-6pt] 
	2019 &  Nicolas Garneau, Mathieu Godbout, David Beauchemin, Audrey Durand et Luc Lamontagne. \href{https://arxiv.org/abs/1912.01706}{\textit{A Robust Self-Learning Method for Fully Unsupervised Cross-Lingual Mappings of Word Embeddings: Making the Method Robustly Reproducible as Well}}. REPORLANG@LREC2020.
		\end{tabular}
		
		\subsection*{\hspace{.5cm} Non scientifique}
		
		\begin{tabular}{L!{\VRule}R}
			2017 & David Beauchemin et Vincent Goulet \textit{\href{https://www.tug.org/TUGboat/Contents/contents38-3.html}{Typesetting actuarial symbols easily and consistently with actuarialsymbol and actuarialangle}}
		\end{tabular}
		
		\subsection*{\hspace{.5cm} Paquetages}
		
		\begin{tabular}{L!{\VRule}R}
			2020 & Marouane Yassine et David Beauchemin \textit{\href{https://deepparse.org}{Deepparse: A state-of-the-art deep learning multinational addresses parser}}\\
			&\\[-6pt] 
			2019 & David Beauchemin \textit{\href{http://notificationdoc.ca/}{Notif: The notification package for every Python project}}\\
			&\\[-6pt] 
			2017 & Simon-Pierre Gadoury et David Beauchemin \textit{\href{https://cran.r-project.org/web/packages/nCopula/index.html}{nCopula: Hierarchical Archimedean Copulas Through Multivariate Compound Distributions}} \\
			&\\[-6pt]  
			2017 & Vincent Goulet et David Beauchemin \textit{\href{http://ctan.org/pkg/actuarialsymbol}{actuarialsymbol - Actuarial notation with \LaTeX}}
		\end{tabular}
	
		\newpage
		
		
		\section*{Communications}
		\subsection*{\hspace{.5cm} Ateliers pratiques}
		
		\begin{tabular}{L!{\VRule}R}
			
			2019 & \textbf{\href{http://raquebec.ulaval.ca/2019/event/les-tests-automatises-en-r}{Les tests automatisés en R}}\\
			& R à Québec\\
			&\\[-6pt]
			2019 & \textbf{Extraction d'information à partir du Web}\\
			& Actulab\\
			&\\[-6pt]
			2019 & \textbf{Atelier pratique en \faGit}\\
			& Meetup Machine Learning Québec\\
			&\\[-6pt]
			2017 & \textbf{\href{https://vigou3.github.io/raquebec-atelier-introduction-r/}{Formation d'introduction à R}}\\
			& R à Québec\\
			&\\[-6pt]
			2017 & \textbf{\href{https://davebulaval.github.io/R_Markdown/}{R \& Markdown}}\\
			& École d'actuariat
		\end{tabular}
				
		\subsection*{\hspace{.5cm} Conférences}	
		\begin{tabular}{L!{\VRule}R}
			2020  & \textbf{La reproductibilité en apprentissage automatique}\\
			&  Webinaires de l'IID\\
			&  Université Laval, Canada \\
			&\\[-6pt]
			2020  & \textbf{Detection of Duplicates Among Non-structured Data From Different Data Sources}\\
			&  Séminaires IID\\
			&  Université Laval, Canada \\
			&\\[-6pt]
			2020  & \textbf{Intelligibility of digital court records: the criminal docket experience}\\
			&  Laboratoire de cyberjustice\\
			&\\[-6pt]
			2020  & \textbf{Détection de doublon parmi des informations non structurées provenant de sources de données externes}\\
			&  Atelier scientifique Intact\\
			&  Université Laval, Canada \\
			&\\[-6pt]
			2019 & \textbf{\href{https://davebulaval.github.io/bonnes-pratiques-git-material/}{Bonnes pratiques \& \faGit}}\\
			& IFT-6010 \\
			& Université Laval, Canada\\
			&\\[-6pt]
			2019 & \textbf{\href{http://raquebec.ulaval.ca/2019/event/lassurance-qualite-et-le-calcul-scientifique}{Assurance qualité, calcul scientifique \& R}}\\
			& R à Québec \& \href{https://www.ulaval.ca/les-etudes/chaires-de-leadership-en-enseignement-cle/les-chaires-de-leadership-en-enseignement/sciences-et-developpement-durable.html}{CLESSN}\\
			& Université Laval, Canada\\
			&\\[-6pt]
			2019  & \textbf{Information gathering using external unstructured data sources}\\
			&  Atelier scientifique Intact\\
			&  Université Laval, Canada \\
			&\\[-6pt]
			2019  & \textbf{Classification de doublons de risques commerciaux}\\
			&  Atelier scientifique Intact\\
			&  Université Laval, Canada \\
			&\\[-6pt]
			2019 & \textbf{Introduction à Scikit-learn}\\
			& École d'été des stagiaires du Groupe de recherche en apprentissage automatique de l'Université Laval\\
			& Université Laval, Canada\\
			&\\[-6pt]
			2018  & \textbf{Système de gestion des bénévoles Agapè}\\
			&  GLO-7035\\
			&  Université Laval, Canada \\
			&\\[-6pt]
			2017  & \textbf{\href{https://github.com/davebulaval/Actulab_COOP}{Où sont les clients que nous ciblons?}}\\
			&  \href{http://www.actulab.ca}{Actulab}\\
			&  Université Laval, Canada
		\end{tabular}
	
		\subsection*{\hspace{.5cm} Article de blogue}
		\begin{tabular}{L!{\VRule}R}
			2020  & \textbf{Training a Recurrent Neural Network (RNN) Using PyTorch}\\
			&  \href{https://www.dotlayer.org/en/blog/2020-08-19-train-a-sequence-model-with-poutyne/machine-learning/}{.Layer}
		\end{tabular}
		
		\section*{Bourses}
		\begin{tabular}{L!{\VRule}R}
			2022 -- \tab[.7cm] & \textbf{Bourse de recherche FQRNT} \\
			& Synthèse, reformulation et explication automatiques des contrats d'assurance \\ 
			&\\[-6pt]
			2020 -- \tab[.7cm] & \textbf{Bourse de recherche} \\
			& Forage de données d'assurance : techniques, éthiques, et sécurité\\
			&\\[-6pt]
			2020 & \textbf{Bourse d'admission} \\
			& Bourse d'admission au doctorat\\
			&\\[-6pt]
			2019 -- 2020 & \textbf{Bourse de recherche} \\
			& Chaire de recherche industrielle CRSNG-Intact Corporation financière sur l'apprentissage automatique en assurance\\
			& Développement logiciel de la librairie Poutyne pour l'apprentissage profond\\
			&\\[-6pt]
			2019 & \textbf{Bourse de recherche} \\
			& Chaire de recherche industrielle CRSNG-Intact Corporation financière sur l'apprentissage automatique en assurance\\
			& Fusion d'enregistrements de risques commerciaux\\
			&\\[-6pt]  
			2018 & \textbf{Bourse d'excellence Gaston Paradis} \\
			& Université Laval\\
			& Implication sociale\\
			&\\[-6pt]
			2018 & \textbf{Bourse Yves-Roy} \\
			& Université Laval\\
			& Meilleur projet en génie logiciel orienté-objet   
		\end{tabular}
		\vspace{10pt}
		\section*{Prix et distinction}
		\begin{tabular}{L!{\VRule}R}
			2016 -- 2018 & \textbf{Gala du mérite étudiant} \\
			& AESGUL - Association des Étudiants en Sciences et Génie de l'Université Laval\\
			& Implication sociale       
		\end{tabular}
	
		\vspace{10pt}
		
		\section*{Implications}
		\subsection*{\hspace{.5cm} Comité}
		\begin{tabular}{L!{\VRule}R}
			2020 -- 2021 & \textbf{Comité d'organisation de R à Québec}\\
			& Président du comité d'organisation des ateliers pratiques \\
			& Université Laval, Canada\\
			&\\[-6pt]
			2020 -- \tab[.7cm] & \textbf{Comité éditorial du blogue}\\
			& Président \\
			& \href{https://www.dotlayer.org/}{.Layer}\\
			&\\[-6pt]
			2020 -- \tab[.7cm] & \textbf{Comité de développement de \textit{Data Universal Tool}}\\
			& Président\\
			& \href{https://www.dotlayer.org/}{.Layer}\\
			&\\[-6pt]
			2020 & \textbf{Atelier de co-design prospectif en IA}\\
			& Participant \\
			& OBVIA - Observatoire international sur les impacts sociétaux de l'IA et du numérique\\
			&\\[-6pt]
			2019 -- 2020 & \textbf{Comité d'amélioration des évaluations d'enseignement}\\
			& Participant \\
			& Faculté des sciences et du génie, Université Laval\\
			&\\[-6pt]
			2018 & \textbf{Comité d'organisation de la semaine de l'apprentissage automatique en assurance}\\
			& Participant \\
			& Meetup Machine Learning Québec
		\end{tabular}
		
		\subsection*{\hspace{.5cm} Jury}
		\begin{tabular}{L!{\VRule}R}
			2018 -- 2020 & \textbf{Membre du jury}\\
			& Mon Travail pratique en 180 secondes \\
			& École d'actuariat, Université Laval\\
			&\\[-6pt]
			2019 & \textbf{Membre du jury}\\
			& Présentation des projets de session du cours GLO-4035/GLO-7035\\
			& Département d'informatique et de génie logiciel, Université Laval
		\end{tabular}
		
		\subsection*{\hspace{.5cm} Coopérative}
		\begin{tabular}{L!{\VRule}R}
			2020 -- \tab[.7cm]  & \textbf{\href{https://baseline.quebec/}{Baseline}} \\
			\multirow{1}{*}{\includegraphics[scale=0.075]{images/Baseline.png} \tab[1.1cm]} & Président du conseil d'administration de la coopérative de travailleurs Baseline en intelligence artificielle\\
			& Québec, Canada
		\end{tabular}

		\subsection*{\hspace{.5cm} Organisme à but non lucratif}
		\begin{tabular}{L!{\VRule}R}
			2021 -- \tab[.7cm]  & \textbf{\href{https://www.dotlayer.org/}{.Layer}} \\
			\multirow{1}{*}{\includegraphics[scale=0.055]{images/DotLayer.png} \tab[1cm]}& Président du conseil d'administration \\
			& Québec, Canada\\
			&\\
			&\\[-6pt]   
			2019 -- 2021  & \textbf{\href{https://www.dotlayer.org/}{.Layer}} \\
			\multirow{1}{*}{\includegraphics[scale=0.055]{images/DotLayer.png} \tab[1cm]}& Vice-président au développement des affaires\\
			& Québec, Canada\\
			&\\
			&\\[-6pt]                      
			2016 -- 2018 & \textbf{AÉACT - Association des étudiants en actuariat de l'Université Laval}\\
			& Université Laval, Canada
		\end{tabular}
		
		\subsection*{\hspace{.5cm} Balado}
		\begin{tabular}{L!{\VRule}R}
			2020 -- \tab[.7cm] & \textbf{IA \& café}\\
			&Animateur invité au balado IA \& café
		\end{tabular}
	\end{document}