
% Document class and font size
%-----------------------------------------------
\documentclass[10pt, oneside]{article}

% Aesthetics and colors
%-----------------------------------------------
\usepackage[hmargin = 1.25cm, vmargin = 1.5cm]{geometry} 
\usepackage[utf8]{inputenc}  % for the portuguese characters 
\usepackage[none]{hyphenat} % no hyphenated words
\usepackage[usenames,dvipsnames]{xcolor} 
\usepackage{microtype} % to optimize spacing
\usepackage{array}
\usepackage{lmodern}
\usepackage{tikz} % For photo placement
\usepackage{fancyhdr}
\pagestyle{fancy}
\fancyhf{}
\newcommand\tab[1][1cm]{\hspace*{#1}}
\usepackage{multirow}
\usepackage{longtable}

\usepackage{fontawesome5}
\usepackage{fontawesome}

% Fonts and tweaks for XeLaTeX
%-----------------------------------------------
\usepackage{fontspec,xltxtra}
\usepackage{xltxtra}
\usepackage{xunicode}
\defaultfontfeatures{Mapping=tex-text}
%\setsansfont[Scale=MatchLowercase,Mapping=tex-text]{Gill Sans} % Font my name at the top


%%% Couleurs
\definecolor{baseline-gray}{HTML}{f0f2f2}
\definecolor{baseline-black}{HTML}{212222}
\definecolor{baseline-blue}{HTML}{2d3bff}
\definecolor{baseline-red}{HTML}{ff5043}

% url color and font
%-----------------------------------------------
\usepackage[colorlinks,urlcolor=baseline-black,bookmarks=false,hypertexnames=true]{hyperref}
\urlstyle{same} 

% Customized section headers
%-----------------------------------------------
\usepackage{titlesec} % Allows creating custom \section's
\titleformat{\section}{\bf\color{baseline-black}
	\scshape\Large\raggedright}{}{0em}{}[\color{black}\titlerule]
\titlespacing{\section}{0pt}{0pt}{5pt} % Spacing around titles
\renewcommand{\headrulewidth}{0pt} % Get rid of the default rule in the header

%contact info
\newcommand{\cvemail}{david.beauchemin.5@ulaval.ca}
\newcommand{\cvphonenumber}{(514) 250-3616}
\newcommand{\cvlinkedin}{https://www.linkedin.com/in/david-beauchemin/?locale=en_US}
\newcommand{\github}{https://github.com/davebulaval}

% Customized subsection headers
%-----------------------------------------------
\titleformat{\subsection}{\bfseries\large\raggedright}{}{0em}{}[\color{black}]
\titlespacing{\subsection}{0pt}{2.5pt}{2.5pt}

% Settings for the vertical timeline
%-----------------------------------------------
\newcolumntype{L}{>{\raggedleft}p{0.14\textwidth}}
\newcolumntype{R}{p{0.8\textwidth}}
\newcommand\VRule{\color{baseline-gray}\vrule width 0.5pt}

\newcommand{\CC}{C\nolinebreak\hspace{-.05em}\raisebox{.4ex}{\tiny\bf +}\nolinebreak\hspace{-.10em}\raisebox{.4ex}{\tiny\bf +}}
\def\CC{{C\nolinebreak[4]\hspace{-.05em}\raisebox{.4ex}{\tiny\bf ++}}}

\begin{document}
	
	% Name in the top center
	%-----------------------------------------------
	\par{\centering{\bf\sffamily\Huge David Beauchemin}\\\vspace{2pt}
		
		
		% Colored boxes with personal info
		%----------------------------------------------- 
		
		\par{\centering
			\href{mailto:\cvemail}{\faEnvelopeO} \,\, \href{mailto:\cvemail}{\cvemail}\\
			\faPhone \,\, \cvphonenumber \\
			\href{\github}{\faGithub}\,\, \href{\github}{See GitHub} \\
			\href{\cvlinkedin}{\faLinkedinSquare}\,\, \href{\cvlinkedin}{See profile}\\
			Fluent in \faPython\ \faRProject\ \faGit\ \LaTeX\ \faLinux \\
			Familiar in \faJava\ \CC \ Bash/ZSH
		}
		
		\vspace{10pt}
		
		\section*{Education}
		
		\begin{tabular}{L!{\VRule}R}
			2020 -- \tab[.7cm] & \textbf{Doctor of Philosophy, Computer Science | Machine Learning | Natural Language Processing}\\
			\multirow{1}{*}{\includegraphics[scale=0.1]{images/UL_P.pdf} \tab[0.6cm]} &  Laval University \\
			& \href{https://graal.ift.ulaval.ca/}{GRAIL - Group for Research in Artificial Intelligence of Laval University}\\
			&\\[-6pt]
			
			2018 -- 2020           & \textbf{Master's degree in Computer Science | Machine Learning | Natural Language Processing}\\
			\multirow{1}{*}{\includegraphics[scale=0.1]{images/UL_P.pdf} \tab[0.6cm]}  &  Laval University \\
			& \href{https://graal.ift.ulaval.ca/}{GRAIL - Group for Research in Artificial Intelligence of Laval University}\\
			& Thesis subject: \textit{Détection de doublons parmi des informations non structurées provenant de sources de données externes}\\
			&\\[-6pt]
			
			2015 -- 2018           & \textbf{Bachelor's degree in actuarial sciences} \\
			\multirow{1}{*}{\includegraphics[scale=0.1]{images/UL_P.pdf} \tab[0.6cm]}  &  Laval University
		\end{tabular}
		
		\vspace{10pt}
		
		% Experience
		%-----------------------------------------------
		\section*{Experiences}
		
		\subsection{Professional}
		\begin{tabular}{L!{\VRule}R}
			2019 -- \tab[.7cm]   &{\bf {Applied AI Researcher in Natural Language Processing}}\\
			\multirow{1}{*}{\includegraphics[scale=0.075]{images/Baseline.png} \tab[1.1cm]}					& Baseline\\
			& Applied AI research for clients \& specialized formation in AI\\
			&\\[-6pt]
			2019 -- \tab[.7cm]   &{\bf {Developer}}\\
			\multirow{1}{*}{\includegraphics[scale=0.15]{images/poutyne.png} \tab[0.18cm]}& \href{https://poutyne.org/}{Poutyne}\\
			& Feature developement \\ 
			&\\[-6pt]
			2019 -- \tab[.7cm]   &{\bf {Founder, Researcher, and Interviewer}}\\
			\multirow{1}{*}{\includegraphics[scale=0.025]{images/OpenLayer.png} \tab[0.8cm]}& \href{https://www.youtube.com/channel/UCB3tYpZ1ojiqAroyDN05Cyw}{OpenLayer podcas}t\\
			& AI content creation\\ 
			&\\[-6pt]
			2018 -- \tab[.7cm]   &{\bf {Event planning}}\\
			\multirow{1}{*}{\includegraphics[scale=0.025]{images/meetup.png} \tab[0.8cm]}& Meetup Machine Learning Quebec\\
			& Co-organization of events in Quebec city \\ 
			&\\[-6pt]
			2019 -- 2020  &{\bf {AI scientist}}\\
			\multirow{1}{*}{\includegraphics[scale=0.09]{images/Vooban.png} \tab[1.15cm]}& Vooban\\
			& Technical advisor in artificial intelligence
		\end{tabular}
		
		\subsection{Teaching}
		\begin{tabular}{L!{\VRule}R}
			2016 -- \tab[.7cm] &{\bf {Teaching Assistant}}\\
			\multirow{1}{*}{\includegraphics[scale=0.1]{images/UL_P.pdf} \tab[0.6cm]}  &  Laval University, École d'actuariat \\
			& \textit{Introduction à l'actuariat I, Gestion du risque financier I, Analyse et traitement collectif du risque, Informatique pour actuaire, Méthodes numériques en actuariat, Mathématiques actuarielles IARD~II, Modèles linéaires en actuariat, Programmation avec R pour l'analyse de données} \\ 
			&\\[-6pt]
			2019 -- 2020&{\bf {Teaching Assistant}}\\
			\multirow{1}{*}{\includegraphics[scale=0.1]{images/UL_P.pdf} \tab[0.6cm]}  &  Laval University, BDRC - Big Data Research Center\\
			& Winter school in machine learning
		\end{tabular}
		
		\subsection{Research}
		\begin{tabular}{L!{\VRule}R}
			2020 -- \tab[.7cm] &{\bf {Research Assistant}}\\
			\multirow{1}{*}{\includegraphics[scale=0.1]{images/UL_P.pdf} \tab[0.6cm]}  &  Laval University, IID - Institute Intelligence and Data \\
			& Female Workers Facing the Challenge of Digital Transformation: A Case Study in the Insurance Sector\\ 
			& Christian Gagné, Future skills centre\\
			2020 &{\bf {Research Assistant}}\\
			\multirow{1}{*}{\includegraphics[scale=0.1]{images/UL_P.pdf} \tab[0.6cm]}  &  Laval University, \textit{École d'actuariat} \\
			& \textit{Plateforme libre de validation et d'étalonnage de prévisions financières et actuarielles}\\ 
			& Vincent Goulet, Laval University actuarial science research chair\\ 
			&\\[-6pt] 
			2019 &{\bf {Research Assistant}}\\
			\multirow{1}{*}{\includegraphics[scale=0.1]{images/UL_P.pdf} \tab[0.6cm]}  &  Laval University, GRAIL - Group for Research in Artificial Intelligence of Laval University \\
			& Information gathering using external unstructured data sources \\ 
			& Luc Lamontagne, NSERC/Intact Financial Corporation Industrial Research Chair in Machine Learning for Insurance\\ 
			&\\[-6pt]
			2019 &{\bf {Research Assistant}}\\
			\multirow{1}{*}{\includegraphics[scale=0.1]{images/UL_P.pdf} \tab[0.6cm]}  &  Laval University, GRAIL - Group for Research in Artificial Intelligence of Laval University \\
			& Learning taxonomy of professional skills \\ 
			& Luc Lamontagne, ENGAGE project\\ 
			&\\[-6pt]
			2018 &{\bf {Research Assistant}}\\
			\multirow{1}{*}{\includegraphics[scale=0.1]{images/UL_P.pdf} \tab[0.6cm]}  &  Laval University, \textit{École d'actuariat} \\
			& Intergenerational equity: metrics for conditional indexation in pension plans \\ 
			& Louis Adam, Laval University actuarial science research chair\\ 
			&\\[-6pt]
			2018 &{\bf {Research Assistant}}\\
			\multirow{1}{*}{\includegraphics[scale=0.1]{images/UL_P.pdf} \tab[0.6cm]}  &  Laval University, GRAIL - Group for Research in Artificial Intelligence of Laval University \\
			& Weighted bootstrapping method for word relation extraction  \\ 
			& Luc Lamontagne
		\end{tabular}
		
		\section*{Publications}
		\subsection*{\hspace{.5cm} Article}
		\begin{tabular}{L!{\VRule}R}
			2020 & David Beauchemin, Nicolas Garneau, Eve Gaumond, Pierre-Luc Déziel, Richard Khoury and Luc Lamontagne. \href{https://arxiv.org/abs/2011.12183}{\textit{Generating Intelligible Plumitifs Summaries: Use Case Application with Ethical Considerations}}. The 13th International Conference on Natural Language Generation. \\
			&\\[-6pt] 
			2020 & Marouane Yassine, David Beauchemin, François Laviolette and Luc Lamontagne. \href{https://arxiv.org/abs/2006.16152}{\textit{Leveraging Subword Embeddings for Multinational Address Parsing}}. IEEE CiSt'20, MNLP. \\
			&\\[-6pt]
			2019 &  Nicolas Garneau, Mathieu Godbout, David Beauchemin, Audrey Durand and Luc Lamontagne. \href{https://arxiv.org/abs/1912.01706}{\textit{A Robust Self-Learning Method for Fully Unsupervised Cross-Lingual Mappings of Word Embeddings: Making the Method Robustly Reproducible as Well}}. REPORLANG@LREC2020.             
		\end{tabular}
		
		\subsection*{\hspace{.5cm} Other publication}
		\begin{tabular}{L!{\VRule}R}
			2018 & David Beauchemin \textit{\href{https://corpus.ulaval.ca/jspui/handle/20.500.11794/67747}{Détection de doublons parmi des informations non structurées provenant de sources de données différentes}}, Master Thesis\\
			2017 & David Beauchemin and Vincent Goulet \textit{\href{https://www.tug.org/TUGboat/Contents/contents38-3.html}{Typesetting actuarial symbols easily and consistently with actuarialsymbol and actuarialangle}}
		\end{tabular}
		
		\subsection*{\hspace{.5cm} Packages}
		
		\begin{tabular}{L!{\VRule}R}
			2020 & Marouane Yassine et David Beauchemin \textit{\href{https://deepparse.org}{Deepparse: A state-of-the-art deep learning multinational addresses parser}}\\
			&\\[-6pt] 
			2019 & David Beauchemin \textit{\href{http://notificationdoc.ca/}{Notif: The notification package for every python project}}\\
			&\\[-6pt] 
			2017 & Simon-Pierre Gadoury and David Beauchemin \textit{\href{https://cran.r-project.org/web/packages/nCopula/index.html}{nCopula: Hierarchical Archimedean Copulas Through Multivariate Compound Distributions}} \\
			&\\[-6pt]  
			2017 & Vincent Goulet and David Beauchemin \textit{\href{http://ctan.org/pkg/actuarialsymbol}{actuarialsymbol - Actuarial notation with \LaTeX}}
		\end{tabular}
	
	\newpage	
		
		\section*{Communications}
		\subsection*{\hspace{.5cm} Workshops}
		
		\begin{tabular}{L!{\VRule}R}
			2019 & \textbf{\href{http://raquebec.ulaval.ca/2019/event/les-tests-automatises-en-r}{\textit{Les tests automatisés en R}}}\\
			& \textit{R à Québec}\\
			&\\[-6pt]
			2019 & \textbf{\textit{Extraction d'information à partir du Web}}\\
			& Actulab\\
			&\\[-6pt]
			2019 & \textbf{\textit{Atelier pratique en \faGit}}\\
			& Meetup Machine Learning Quebec\\
			&\\[-6pt]
			2017 & \textbf{\href{https://vigou3.github.io/raquebec-atelier-introduction-r/}{\textit{Formation d'introduction à R}}}\\
			& \textit{R à Québec}\\
			&\\[-6pt]
			2017 & \textbf{\href{https://davebulaval.github.io/R_Markdown/}{R \& Markdown}}\\
			& \textit{École d'actuariat} \\
			&\\[-6pt]
		\end{tabular}
		
		\subsection*{\hspace{.5cm} Presentations}
		
		\begin{tabular}{L!{\VRule}R}
			2020 -- 2021  & \textbf{\textit{La reproductibilité en apprentissage automatique}}\\
			&  Webinaires de l'IID \& bootcamp de l'IID\\
			&  Laval University, Canada \\
			&\\[-6pt]
			2020  & \textbf{Detection of Duplicates Among Non-structured Data From Different Data Sources}\\
			&  Séminaires IID\\
			&  Université Laval, Canada \\
			&\\[-6pt]
			2020  & \textbf{Intelligibility of digital court records: the criminal docket experience}\\
			&  Cyberjustice laboratory\\
			&\\[-6pt]
			2020  & \textbf{\textit{Détection de doublons parmi des informations non structurées provenant de sources de données externes}}\\
			&  Intact scientific workshop\\
			&  Laval University, Canada \\
			&\\[-6pt]
			2019 & \textbf{\href{https://davebulaval.github.io/bonnes-pratiques-git-material/}{\textit{Bonnes pratiques \& \faGit}}}\\
			& IFT-6010 \\
			& Laval University, Canada\\
			&\\[-6pt]
			2019 & \textbf{\href{http://raquebec.ulaval.ca/2019/event/lassurance-qualite-et-le-calcul-scientifique}{\textit{Assurance Qualité, calcul scientifique \& R}}}\\
			& R à Québec \& \href{https://www.ulaval.ca/les-etudes/chaires-de-leadership-en-enseignement-cle/les-chaires-de-leadership-en-enseignement/sciences-et-developpement-durable.html}{CLESSN}\\
			& Laval University, Canada\\
			&\\[-6pt]
			2019  & \textbf{Information gathering using external unstructured data sources}\\
			&  Intact scientific workshop\\
			&  Laval University, Canada \\
			&\\[-6pt]
			2019  & \textbf{\textit{Classification de doublons de risques commerciaux}}\\
			&  Intact scientific workshop\\
			&  Laval University, Canada \\
			&\\[-6pt]
			2019 & \textbf{\textit{Introduction à Scikit-learn}}\\
			& Summer school of the Group for Research in Artificial Intelligence of Laval University interns\\
			& Laval University, Canada\\
			&\\[-6pt]
			2018  & \textit{\textbf{Système de gestion des bénévoles Agapè}}\\
			&  GLO-7035\\
			&  Laval University, Canada \\
			&\\[-6pt]
			2017  & \textbf{\href{https://github.com/davebulaval/Actulab_COOP}{\textit{Où sont les clients que nous ciblons?}}}\\
			&  \href{http://www.actulab.ca}{Actulab}\\
			&  Laval University, Canada
		\end{tabular}
		\subsection*{\hspace{.5cm} Blog Article}
		\begin{tabular}{L!{\VRule}R}
			2020  & \textbf{Training a Recurrent Neural Network (RNN) Using PyTorch}\\
			&  \href{https://www.dotlayer.org/en/blog/2020-08-19-train-a-sequence-model-with-poutyne/machine-learning/}{.Layer}
		\end{tabular}
		
	\newpage
		
		\section*{Funding and Scholarships}
		\begin{tabular}{L!{\VRule}R}
			2020 -- \tab[.7cm] & \textbf{Research Grant} \\
			& \textit{Forage de données d'assurance : techniques, éthiques, et sécurité}\\
			&\\[-6pt]
			2020 & \textbf{Entrance Scholarship} \\
			& Doctoral Entrance Scholarship\\
			&\\[-6pt]	
			2019 -- 2020 & \textbf{Research Grant} \\
			& NSERC/Intact Financial Corporation Industrial Research Chair in Machine Learning for Insurance\\
			& Software engineering development of Poutyne deep learning framework\\
			&\\[-6pt]
			2019 & \textbf{Research Grant} \\
			& NSERC/Intact Financial Corporation Industrial Research Chair in Machine Learning for Insurance\\
			& Information gathering using external data sources\\
			&\\[-6pt]  
			2018 & \textbf{Gaston Paradis Excellence Award} \\
			& Laval University\\
			& Social involvement\\
			&\\[-6pt]
			2018 & \textbf{Yves-Roy Award} \\
			& Laval University\\
			& Best software engineering term project      
		\end{tabular}
		
		\vspace{10pt}
		\section*{Award and Distinctions}
		
		\begin{tabular}{L!{\VRule}R}
			2016 -- 2018 & \textbf{Gala for the student excellence} \\
			& AESGUL (Association des Étudiants en Sciences et Génie de l'Université Laval)\\
			& Social involvement      
		\end{tabular}
		\vspace{10pt}
		
		\section*{Community Involvement}
		\subsection*{\hspace{.5cm} Committee}
		\begin{tabular}{L!{\VRule}R}
			2020 -- \tab[.7cm] & \textbf{R à Québec Organization Committee}\\
			& President of organization of the workshop sessions \\
			& Laval University, Canada\\
			&\\[-6pt]
			2020 -- \tab[.7cm] & \textbf{Editorial Blog Committee}\\
			& President \\
			& \href{https://www.dotlayer.org/}{.Layer}\\
			&\\[-6pt]
			2020 -- \tab[.7cm] & \textbf{Committee Data Universal Tool Development}\\
			& President \\
			& \href{https://www.dotlayer.org/}{.Layer}\\
			&\\[-6pt]
			2020 & \textbf{Co-design AI Prospective Workshop}\\
			& Participant \\
			& IOSIAIDT - International Observatory on the Societal Impacts of Artificial Intelligence and Digital Technology\\
			&\\[-6pt]
			2019 -- 2020 & \textbf{Committee for the improvement of teaching evaluations}\\
			&Participant\\
			& Faculty of Science and Engineering, Laval University\\
			&\\[-6pt]
			2018 & \textbf{Organisational committee for Machine Learning in Insurance Week}\\
			&Participant\\
			& Meetup Machine Learning Quebec
		\end{tabular}
		
		\subsection*{\hspace{.5cm} Jury}
		\begin{tabular}{L!{\VRule}R}
			2018 -- 2020 & \textbf{Member of the Jury}\\
			& My assignment in 180 seconds \\
			& \textit{École d'actuariat}, Laval University\\
			&\\[-6pt]
			2019 -- 2020 & \textbf{Member of the Jury}\\
			& Presentation of the assignment\\
			& Department of Computer Science and Software Engineering, Laval University
		\end{tabular}
		
		\subsection*{\hspace{.5cm} Cooperative}
		\begin{tabular}{L!{\VRule}R}
			2020 -- \tab[.7cm]  & \textbf{\href{https://baseline.quebec/}{Baseline}} \\
			\multirow{1}{*}{\includegraphics[scale=0.075]{images/Baseline.png} \tab[1.1cm]} & Chairman of the Board of Directors of Baseline, a worker cooperative in artificial intelligence\\
			& Quebec, Canada
		\end{tabular}

		\subsection*{\hspace{.5cm} Non-Profit Associations}
		\begin{tabular}{L!{\VRule}R}
			2021 -- \tab[.7cm]  & \textbf{\href{https://www.dotlayer.org/}{.Layer}} \\
			\multirow{1}{*}{\includegraphics[scale=0.055]{images/DotLayer.png} \tab[1cm]}& President\\
			& Quebec, Canada\\
			&\\
			2019 -- 2021 & \textbf{\href{https://www.dotlayer.org/}{.Layer}} \\
			\multirow{1}{*}{\includegraphics[scale=0.055]{images/DotLayer.png} \tab[1cm]}& Vice-president of business development\\
			& Quebec, Canada\\
			&\\
			&\\[-6pt]                      
			2016 -- 2018 & \textbf{AÉACT (Association des Étudiants en Actuariat de l'Université Laval)}\\
			& Laval University, Canada
		\end{tabular}
	
		
		\subsection*{\hspace{.5cm} Podcast}
		\begin{tabular}{L!{\VRule}R}
			2020 -- \tab[.7cm] & \textbf{IA \& café}\\
			&Invited co-host of the podcast \textit{IA \& café}
		\end{tabular}
	\end{document}