\documentclass[10pt, oneside]{article}

% Aesthetics and colors
%-----------------------------------------------
\usepackage[hmargin = 1.25cm, vmargin = 1.5cm]{geometry} 
\usepackage[utf8]{inputenc}  % for the portuguese characters 
\usepackage[none]{hyphenat} % no hyphenated words
\usepackage[usenames,dvipsnames]{xcolor} 
\usepackage{microtype} % to optimize spacing
\usepackage{array}
\usepackage{lmodern}
\usepackage{tikz} % For photo placement
\usepackage{fancyhdr}
\pagestyle{fancy}
\fancyhf{}
\newcommand\tab[1][1cm]{\hspace*{#1}}
\usepackage{multirow}
\usepackage{longtable}

\usepackage{fontawesome}

% Fonts and tweaks for XeLaTeX
%-----------------------------------------------
\usepackage{fontspec,xltxtra}
\usepackage{xltxtra}
\usepackage{xunicode}
\defaultfontfeatures{Mapping=tex-text}
%\setsansfont[Scale=MatchLowercase,Mapping=tex-text]{Gill Sans} % Font my name at the top


%%% Couleurs
\definecolor{baseline-gray}{HTML}{f0f2f2}
\definecolor{baseline-black}{HTML}{212222}
\definecolor{baseline-blue}{HTML}{2d3bff}
\definecolor{baseline-red}{HTML}{ff5043}

% url color and font
%-----------------------------------------------
\usepackage[colorlinks,urlcolor=baseline-black,bookmarks=false,hypertexnames=true]{hyperref}
\urlstyle{same} 

% Customized section headers
%-----------------------------------------------
\usepackage{titlesec} % Allows creating custom \section's
\titleformat{\section}{\bf\color{baseline-black}
	\scshape\Large\raggedright}{}{0em}{}[\color{black}\titlerule]
\titlespacing{\section}{0pt}{0pt}{5pt} % Spacing around titles
\renewcommand{\headrulewidth}{0pt} % Get rid of the default rule in the header

%contact info
\newcommand{\cvemail}{david.beauchemin.5@ulaval.ca}
\newcommand{\cvphonenumber}{(514) 250-3616}
\newcommand{\cvlinkedin}{https://www.linkedin.com/in/david-beauchemin/}
\newcommand{\github}{https://github.com/davebulaval}

\newcommand{\guillemet}[1]{\guillemotleft #1 \guillemotright}

\newcommand{\link}[2]{\href{#1}{#2~{\smaller\faExternalLink*}}}

% Customized subsection headers
%-----------------------------------------------
\titleformat{\subsection}{\bfseries\large\raggedright}{}{0em}{}[\color{black}]
\titlespacing{\subsection}{0pt}{2.5pt}{2.5pt}

% Settings for the vertical timeline
%-----------------------------------------------
\newcolumntype{L}{>{\raggedleft}p{0.14\textwidth}}
\newcolumntype{R}{p{0.8\textwidth}}
\newcommand\VRule{\color{baseline-gray}\vrule width 0.5pt}

\usepackage{fontawesome}

%---------------------------------------------------------
% ----- Begginning of the documment ---------------
%---------------------------------------------------------
\begin{document}
	
	% Name in the top center
	%-----------------------------------------------
	\par{\centering{\bf\sffamily\Huge David Beauchemin}\\\vspace{2pt}
		
		
		% Colored boxes with personal info
		%----------------------------------------------- 
		
		\par{\centering
			\href{mailto:\cvemail}{\faEnvelopeO} \,\, \href{mailto:\cvemail}{\cvemail}\\
			\faPhone \,\, \cvphonenumber \\
			\href{\github}{\faGithub}\,\, \href{\github}{Voir GitHub} \\
			\href{\cvlinkedin}{\faLinkedinSquare}\,\, \href{\cvlinkedin}{Voir profil}\\
		}
		
		\vspace{10pt}
		
		% Formal education
		%-----------------------------------------------
		\section*{Études}
		
		\begin{tabular}{L!{\VRule}R}
			2020 -- \tab[.7cm] & \textbf{Doctorat en informatique | Apprentissage automatique | Traitement automatique du langage naturel}\\
			\multirow{1}{*}{\includegraphics[scale=0.1]{images/UL_P.pdf} \tab[0.6cm]} &  Université Laval \\
			& \href{https://graal.ift.ulaval.ca/}{GRAAL - Groupe de recherche en apprentissage automatique de l'Université Laval}\\
			&\\[-6pt]
			
			2018 -- 2020           & \textbf{Maîtrise en informatique | Apprentissage automatique | Traitement automatique du langage naturel}\\
			\multirow{1}{*}{\includegraphics[scale=0.1]{images/UL_P.pdf} \tab[0.6cm]}  &  Université Laval \\
			& \href{https://graal.ift.ulaval.ca/}{GRAAL - Groupe de recherche en apprentissage automatique de l'Université Laval}\\
			& Sujet mémoire: \textit{Détection de doublons parmi des informations non structurées provenant de sources de données externes}\\
			&\\[-6pt]
			
			2015 -- 2018           & \textbf{Baccalauréat en actuariat} \\
			\multirow{1}{*}{\includegraphics[scale=0.1]{images/UL_P.pdf} \tab[0.6cm]}  &  Université Laval
		\end{tabular}
		
		\vspace{10pt}
		
		% Experience
		%-----------------------------------------------
		\section*{Expériences}
		
		\subsection{Professionnelles}
		\begin{tabular}{L!{\VRule}R}
			2020 -- \tab[.7cm] &{\bf {Membre fondateur, expert IA et Directeur général (depuis 2022)}}\\
			\multirow{1}{*}{\includegraphics[scale=0.075]{images/baseline.png} \tab[1.1cm]}& \href{https://www.baseline.quebec}{Coopérative de solidarité Baseline en intelligence artificielle}\\
			& \\ 
			&\\[-6pt]
			2019 -- \tab[.7cm] &{\bf {Fondateur, recherchiste, et intervieweur}}\\
			\multirow{1}{*}{\includegraphics[scale=0.025]{images/OpenLayer.png} \tab[0.8cm]}& \href{https://www.youtube.com/channel/UCB3tYpZ1ojiqAroyDN05Cyw}{OpenLayer podcast}\\
			& Création de contenu en IA avec plus de 30 000 vues sur une cinquantaine d'épisodes\\ 
			&\\[-6pt]
			2018 -- \tab[.7cm]   &{\bf {Organisation d'événements}}\\
			\multirow{1}{*}{\includegraphics[scale=0.025]{images/meetup.png} \tab[0.8cm]}& Meetup Machine Learning Québec\\
			& Co-organisation des différents événements à Québec \\ 
			&\\[-6pt]
			2019 -- 2020   &{\bf {Développeur}}\\
			\multirow{1}{*}{\includegraphics[scale=0.15]{images/poutyne.png} \tab[0.18cm]}& \href{https://poutyne.org/}{Poutyne}\\
			& Développement de fonctionnalité \\ 
			&\\[-6pt]
			2019 -- 2020  &{\bf {Scientifique IA}}\\
			\multirow{1}{*}{\includegraphics[scale=0.09]{images/Vooban.png} \tab[1.15cm]}& Vooban\\
			& Conseiller technique en intelligence artificielle
		\end{tabular}
		
		\subsection{Enseignements}
		\begin{tabular}{L!{\VRule}R}
			2022 -- \tab[.7cm] &{\bf {Auxiliaire d'enseignement}}\\
			\multirow{1}{*}{\includegraphics[scale=0.1]{images/UL_P.pdf} \tab[0.6cm]}  &  Université Laval, Département d'informatique et de génie logiciel \\
			& Base de données avancées (GLO-4035/GLO-7035)\\ 
			2016 -- 2022 &{\bf {Auxiliaire de cours}}\\
			\multirow{1}{*}{\includegraphics[scale=0.1]{images/UL_P.pdf} \tab[0.6cm]}  &  Université Laval, École d'actuariat \\
			& Introduction à l'actuariat I, Gestion du risque financier I, Analyse et traitement collectif du risque, Informatique pour Laval University, Méthodes numériques en actuariat, Mathématiques actuarielles IARD II, Modèles linéaires en actuariat, Programmation avec R pour l'analyse de données, Correction des rapports de stage \\ 
			&\\[-6pt]
			2019 -- 2020&{\bf {Auxiliaire d'enseignement}}\\
			\multirow{1}{*}{\includegraphics[scale=0.1]{images/UL_P.pdf} \tab[0.6cm]}  &  Université Laval, CRDM - Centre de recherche en données massives \\
			& École d'hiver en apprentissage automatique
		\end{tabular}
		
		\section*{Implications}
		\subsection*{\hspace{.5cm} Comité}
		\begin{tabular}{L!{\VRule}R}
			2020 -- 2021 & \textbf{Comité d'organisation de R à Québec}\\
			& Président du comité d'organisation des ateliers pratiques \\
			& Université Laval, Canada\\
			&\\[-6pt]
			2020 -- \tab[.7cm] & \textbf{Comité éditorial du blogue}\\
			& Président \\
			& \href{https://www.dotlayer.org/}{.Layer}\\
			\end{tabular}
		
		\begin{tabular}{L!{\VRule}R}
			2020 -- \tab[.7cm] & \textbf{Comité de développement de \textit{Data Universal Tool}}\\
			& Président\\
			& \href{https://www.dotlayer.org/}{.Layer}\\
			&\\[-6pt]
			2020 & \textbf{Atelier de co-design prospectif en IA}\\
			& Participant \\
			& OBVIA - Observatoire international sur les impacts sociétaux de l'IA et du numérique\\
			&\\[-6pt]
			2019 -- 2020 & \textbf{Comité d'amélioration des évaluations d'enseignement}\\
			& Participant \\
			& Faculté des sciences et du génie, Université Laval\\
			&\\[-6pt]
			2018 & \textbf{Comité d'organisation de la semaine de l'apprentissage automatique en assurance}\\
			& Participant \\
			& Meetup Machine Learning Québec
		\end{tabular}
		
		\subsection*{\hspace{.5cm} Coopérative}
		\begin{tabular}{L!{\VRule}R}
			2020 -- 2022 & \textbf{\href{https://baseline.quebec/}{Baseline}} \\
			\multirow{1}{*}{\includegraphics[scale=0.075]{images/Baseline.png} \tab[1.1cm]} & Président du conseil d'administration de la coopérative de travailleurs Baseline en intelligence artificielle\\
			& Québec, Canada
		\end{tabular}

		\subsection*{\hspace{.5cm} Organisme à but non lucratif}
		\begin{tabular}{L!{\VRule}R}
			2021 -- \tab[.7cm]  & \textbf{\href{https://www.dotlayer.org/}{.Layer}} \\
			\multirow{1}{*}{\includegraphics[scale=0.055]{images/DotLayer.png} \tab[1cm]}& Président du conseil d'administration \\
			& Québec, Canada\\
			&\\
			&\\[-6pt]   
			2019 -- 2021  & \textbf{\href{https://www.dotlayer.org/}{.Layer}} \\
			\multirow{1}{*}{\includegraphics[scale=0.055]{images/DotLayer.png} \tab[1cm]}& Vice-président au développement des affaires\\
			& Québec, Canada\\
			&\\
			&\\[-6pt]                      
			2016 -- 2018 & \textbf{AÉACT - Association des étudiants en actuariat de l'Université Laval}\\
			& Université Laval, Canada
		\end{tabular}
		
		\subsection*{\hspace{.5cm} Balado}
		\begin{tabular}{L!{\VRule}R}
			2020 -- \tab[.7cm] & \textbf{IA \& café}\\
			&Animateur invité au balado IA \& café
		\end{tabular}
	\end{document}